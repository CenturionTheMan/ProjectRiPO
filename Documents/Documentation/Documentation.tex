\documentclass{article}
\input{TexBase/DocumentBase.tex}

    %! RiPO GoogleDrive Link: https://drive.google.com/drive/folders/1fY_jmIjsITuAyMuPgNjyoRm-0CY-nvYN

    %! Tech-stack
    % - Python 
    %   - OpenCV
    %   - TKinter
    % - Windows
    % - Wymagania sprzętowe > coś napisać
    % - Wszystko lokalne - bezpieczne > widows z antywirusem
    % - Wymagania do kamer
    % - konteneryzacja > Docker


    % ! PLAN PRACY
    % 1. Znaleźć nagrań [Case studies]          :> Zychowicz
    % 2. Wybrać co indetyfikujemy ()            :> Zychowicz
    % 3. [Opis funkcjonalny]                    :> Dziedziak                    
    % 3. [Architektura wysokopoziomowa systemu] :> Zychowicz
    % 4. [Planowany zakres prac rozwojowych]    :> X
    % 4. [Plan testów systemu]                  :> X
    % (kiedykolwiek) [Dobór technologii]        :> Dziedziak %?DONE
    % (??????) [Architektura logiczna systemu]  :> Dziedziak


    \section{Wstęp}
    \subsection{Opis projektu i systemu}
Projekt ma na celu opracowanie systemu obsługującego kamerę samochodową, wyposażoną
w funkcję wykrywania określonych obiektów, z możliwością konfiguracji jej parametrów.
Głównym zadaniem systemu jest analiza obrazu z kamery oraz sygnalizacja użytkownikowi wykrycia
obiektu spełniającego określone kryteria.
W ramach projektu przewiduje się wykrywanie takich obiektów jak: samochody, piesi i znaki drogowe.

W skład systemu wchodzi również edytor umożliwiający konfigurację oprogramowania wykonawczego.
Edytor ten umożliwia kadrowanie obrazu, edycję linii, wybór obiektów do wykrywania oraz ustawianie
parametrów wykrywania obiektów. Dodatkowo, użytkownik może ustawić metodę informowania o wykryciu obiektów,
taką jak ramka wokół obiektu, alert na ekranie lub alert dźwiękowy.

System zapewnia obsługę zarówno nagrania wideo z kamery, jak i obrazu w czasie rzeczywistym.
Dzięki temu użytkownik ma możliwość monitorowania otoczenia pojazdu w czasie rzeczywistym oraz analizy
wcześniej zarejestrowanych nagrań.

\subsection{Analiza istniejących rozwiązań}
\subsubsection*{FineVu GX1000}
\begin{itemize}
    \item Zalety
          \begin{itemize}
              \item Wysoka rozdzielczość kamer: Quad HD + Quad HD
              \item Szeroki kąt widzenia: $122^\circ / 122^\circ$
              \item Funkcja detekcji ruchu
              \item Ostrzeżenia o fotoradarach, kamerach średniej prędkości i innych kontrolach drogowych
              \item Auto Night Vision - tryb nocny i automatyczna poprawa obrazu w dzień
          \end{itemize}
    \item Wady
          \begin{itemize}
              \item Wysoka cena: 1349 zł lub 1399 zł w zaleznosci od rozmiaru karty pamięci
              \item Ryzyko przegrzania: W niektórych warunkach klimatycznych lub przy długotrwałym
                    użytkowaniu istnieje ryzyko przegrzewania się urządzenia, co może wpłynąć na jego wydajność.
              \item Brak funkcji wykrywania statycznych obiektów (pachołki, krawężniki itp.)
          \end{itemize}
\end{itemize}

Link: \textit{\url{https://finevu.pl/kamery-samochodowe/finevu-gx1000/}}


\subsubsection*{M3S/M6S Infrared Night Vision Camera for Cars} %! Trzeba to sprawdzić jeszcze bo dziwne
\begin{itemize}
    \item Zalety
          \begin{itemize}
              \item Technologia termowizyjna: umożliwia on użytkownikom obserwację otoczenia pojazdu
                    w warunkach słabego oświetlenia i nawet w całkowitej ciemności.
              \item Wykrywanie obiektów: Produkt może być używany do wykrywania obiektów na drodze,
                    włączając w to zwierzęta, osoby poruszające się pieszo czy inne pojazdy,
                    co może pomóc w unikaniu kolizji.
          \end{itemize}
    \item Wady
          \begin{itemize}
              \item Niska dostępność
          \end{itemize}
\end{itemize}

Link: \textit{\url{https://www.infiray.com/products/m6s-infrared-night-vision-camera-for-cars.html}}
 % DONE??
        
    \section{Case studies}
    \subsection{Sytuacje obsługiwane przez system}


System będzie mógł przyjmować obraz zarówno z kamery zamontowanej z przodu jak i z tyłu pojazdu, ale tylko z jednej z nich w tym samym czasie.
  
Podstawową funkcją systemu będzie rysowanie statycznych, poziomych linii pomocniczych na obrazie.
  
Parametry tych linii oraz inne parametry obrazu będzie można dopasowywać przy pomocy programu konfiguracyjnego.

Dodatkową funkcją systemu będzie możliwość wykrywania obiektów.

Rodzaje obiektów, które będzie w stanie wykrywać system to:

\begin{itemize}
	\item Samochody
	\item Piesi
	\item Wybrane rodzaje znaków drogowych
	\begin{itemize}
		\item Znaki STOP
		\item Znaki ostrzegawcze
	 \end{itemize}
 \end{itemize}
 
 W przypadku wykrycia jednego z powyższych obiektów system będzie mógł ostrzec o nich kierowcę w sposób wybrany w konfiguratorze. 
 
Przewidywane sposoby ostrzegania to:
\begin{itemize}
 	\item Narysowanie ramki wokół wykrytego obiektu.
 	\item Uruchomienie sygnału dźwiękowego.
 	\item Wyświetlenie na ekranie ostrzeżenia tekstowego.
\end{itemize}
  

  
 % PRZYKŁADY Z DZIAŁANIA (FOTY)

    \section{Opis funkcjonalny}
    \label{funkcjonalne}
System został zaprojektowany w celu skutecznego wykrywania i identyfikacji 
różnych obiektów na drodze, w tym znaków drogowych, pojazdów oraz pieszych. 
Oprócz tego, system oferuje dodatkowe możliwości, które pozwalają na bardziej 
kompleksową obsługę sytuacji drogowych. Poniżej przedstawiamy szczegółowy 
opis funkcjonalny systemu

\begin{itemize}
    \item \textbf{Wykrywanie pojazdów:} System jest zdolny do wykrywania różnych
     typów pojazdów poruszających się na drodze, włączając w to samochody osobowe, 
     ciężarówki oraz inne pojazdy.
    
     \item \textbf{Wykrywanie pieszych:} System jest wyposażony w funkcję wykrywania pieszych 
    poruszających się wzdłuż drogi lub przechodzących przez nią. 
    
    \item \textbf{Wykrywanie ostrzegawczych znaków drogowych:} System umożliwia identyfikację i klasyfikację 
    ostrzegawczych znaków drogowych.

    \item \textbf{Alerty:} System może dawać znać o wykrytych obiektach na różne sposoby, 
    takie jak ramka wokół wykrytego obiektu, czy ostrzeżenie dźwiękowe. 
    
    \item \textbf{Rysowanie lini:} Na ekranie kamery rysowane są linie ułatwiające parkowanie.

    \item \textbf{Konfigurowalne parametry:} Użytkownik ma możliwość konfiguracji parametrów dotyczących 
    wykrywania obiektów, rysowania linii oraz wyświetlania alertów. 
    Może dostosować te parametry do swoich indywidualnych preferencji i potrzeb.

\end{itemize} % Wymagania funkcjonalne

    \section{Architektura wysokopoziomowa systemu}
    \subsection{Opis}
System zawiera następujące elementy:
\begin{itemize}
	\item Kamera przednia
	\item Kamera tylna
	\item Komputer
	\item Ekran
	\item Głośnik
\end{itemize}

Obraz z kamer przesyłany jest do komputera, gdzie zostaje on przetworzony, oraz przeprowadzany jest proces wykrywania obiektów. Następnie komputer przesyła przetworzony obraz na ekran, oraz może uruchomić sygnał dźwiękowy jeśli wykryty zostanie określony typ obiektu.



\subsection{Schemat}
    \begin{figure}[H]
	\centering
	\resizebox{\columnwidth}{!}{%
		\includegraphics{Img/architektura.png}%
	}
	\caption{Schemat architektury wysokopoziomowej systemu:}
	\label{fig:architektura_diagram}
\end{figure} % żerżnąć z jego przykładu i uszczegółowić

    \section{Architektura logiczna systemu}
        \subsection{Opis}
    Architektura logiczna systemu składa się z trzech modułów, które współpracują ze sobą w celu zapewnienia 
    kompleksowej funkcjonalności systemu.

    \begin{itemize}
        \item \textbf{Konfigurator}
        \begin{itemize}
            \item Odpowiada za interakcję użytkownika z systemem poprzez interfejs graficzny. 
            Użytkownik ma możliwość dokonywania modyfikacji ustawień dotyczących działania oprogramowania
            wykonawczego za pomocą interfejsu.
            \item Umożliwia wprowadzanie ustawień takich jak kadrowanie obrazu, 
            edycja linii rysowanych na obrazie, wybór wykrywanych obiektów oraz parametryzacja 
            procesu wykrywania i informowania o obiektach.
        \end{itemize}
        
        \item \textbf{Dane konfiguracyjne}
        \begin{itemize}
            \item Stanowi pośrednią warstwę między konfiguratorem a oprogramowaniem wykonawczym. 
            Jest odpowiedzialny za przechowywanie ustawień konfiguracyjnych.
        \end{itemize}

        \item \textbf{Oprogramowanie wykonawcze}
        \begin{itemize}
            \item Jest odpowiedzialny za przetwarzanie obrazu z kamery 
            oraz realizację funkcji związanych z wykrywaniem i identyfikacją obiektów na drodze.
            \item Podczas działania oprogramowania wykonawczego, na ekranie wyświetlany jest 
            obraz z kamery lub nagranie wraz z elementami graficznymi, takimi jak linie i ostrzeżenia.
        \end{itemize}
    \end{itemize}
    

    \subsection{Schemat}
    \begin{figure}[H]
        \centering
        \resizebox{\columnwidth/2}{!}{%
        \includegraphics{Img/ISO42010_diagram.jpg}%
        }
        \caption{Schemat architektury logicznej systemu:}
        \label{fig:ISO42010_DIAGRAM}
    \end{figure} %  ISO 42010 / https://www.youtube.com/watch?v=AGWdh4GO8Lw

    \section{Dobór technologii}
    W niniejszym punkcie przedstawiamy listę wymagań technicznych dla tworzonego systemu. 
Przed rozpoczęciem implementacji oprogramowania kluczowe jest dokładne określenie środowiska, 
identyfikacja kluczowych komponentów technologicznych oraz uwzględnienie aspektów związanych 
z systemem operacyjnym, infrastrukturą sprzętową, bezpieczeństwem danych oraz wyborem 
technologii programistycznych.

\subsubsection*{System Operacyjny}
Oprogramowanie będzie kompatybilne z systemem operacyjnym Windows.

\subsubsection*{Wymagania Sprzętowe} %! To do research'u jeszcze
\begin{itemize}
    \item Procesor: Intel Core i5 lub równoważny
    \item Pamięć RAM: Minimum 8GB
    \item Karta graficzna: Zintegrowana lub dedykowana wspierająca OpenGL %?????
    \item System operacyjny: Windows 7 lub nowszy
\end{itemize}

\subsubsection*{Podstawowe Predykcje Wydajnościowe}
Oprogramowanie powinno działać płynnie na wymaganym sprzęcie, zapewniając responsywność 
interfejsu użytkownika oraz szybką obróbkę danych w czasie rzeczywistym.

\subsubsection*{Kwestie Sieciowe i Bezpieczeństwa}
\begin{itemize}
    \item Oprogramowanie będzie działać w trybie lokalnym, nie wymagając połączenia z internetem.
    \item System Windows będzie wyposażony w aktualną ochronę antywirusową w celu zapewnienia bezpieczeńwa danych.
\end{itemize}

\subsubsection*{Wymagania dotyczące Kamer}
\begin{itemize}
    \item Oprogramowanie będzie kompatybilne z różnymi modelami kamer kompatybilnymi z systemem Windows.
    \item Wsparcie dla standardowych rozdzielczości obrazu oraz prędkości klatek.
\end{itemize}

\subsubsection*{Język Programowania i Biblioteki}
\begin{itemize}
    \item Oprogramowanie będzie napisane w języku Python.
    \item Do przetwarzania obrazu będzie wykorzystywana biblioteka OpenCV.
    \item Interfejs zostanie zaimplementowany przy użyciu biblioteki TKinter.
\end{itemize}

\subsubsection*{Konteneryzacja}
Oprogramowanie zostanie skonteneryzowane przy użyciu narzędzia Docker, 
aby zapewnić łatwość w wdrażaniu oraz zarządzaniu środowiskiem aplikacyjnym.   % Coś nazmyślać

    \section{Planowany zakres prac rozwojowych}
    
\begin{table}[H]
    \centering
    \begin{tabular}{|ccc|l|}
        \hline
        \multicolumn{3}{|c|}{\textbf{Tydzień}}                                                & \multicolumn{1}{c|}{\multirow{2}{*}{\textbf{Zadania}}}                                                                                                          \\ \cline{1-3}
        \multicolumn{1}{|c|}{\textbf{numer}} & \multicolumn{1}{c|}{\textbf{od}} & \textbf{do} & \multicolumn{1}{c|}{}                                                                                                                                           \\ \hline
        \multicolumn{1}{|c|}{1}              & \multicolumn{1}{c|}{2024-03-25}  & 2024-03-31  & \begin{tabular}[c]{@{}l@{}}- Omówienie   tematu \\ - Poszukiwanie pomysłów do realizacji w ramach projektu. \\ - Wybór narzędzi   i technologii\end{tabular}   \\ \hline
        \multicolumn{1}{|c|}{2}              & \multicolumn{1}{c|}{2024-04-01}  & 2024-04-07  & \begin{tabular}[c]{@{}l@{}}- Uszczegółowianie założeń projektowych\\ - Tworzenie dokumentacji.\end{tabular}                                                     \\ \hline
        \multicolumn{1}{|c|}{3}              & \multicolumn{1}{c|}{2024-04-08}  & 2024-04-14  & - Nauka obsługi wybranych narzędzi.                                                                                                                             \\ \hline
        \multicolumn{1}{|c|}{4}              & \multicolumn{1}{c|}{2024-04-15}  & 2024-04-21  & - Produkcja modelu zdolnego do wykrywania obiektów                                                                                     \\ \hline
        \multicolumn{1}{|c|}{5}              & \multicolumn{1}{c|}{2024-04-22}  & 2024-04-28  & - Produkcja modelu zdolnego do wykrywania obiektów                                                                                  \\ \hline
        \multicolumn{1}{|c|}{6}              & \multicolumn{1}{c|}{2024-04-29}  & 2024-05-05  & - Stworzenie szkieletu aplikacji z funkcjonalnością wykrywania obiektów                                                                                         \\ \hline
        \multicolumn{1}{|c|}{7}              & \multicolumn{1}{c|}{2024-05-06}  & 2024-05-12  & - Implementacja funkcjonalności rysowania linii pomocniczych                                                                                                    \\ \hline
        \multicolumn{1}{|c|}{8}              & \multicolumn{1}{c|}{2024-05-13}  & 2024-05-19  & \begin{tabular}[c]{@{}l@{}}- Ogólna weryfikacja spełnienia założeń projektowych.\\ - Kontrola jakości i poprawki związane z użytkowaniem aplikacji\end{tabular} \\ \hline
        \multicolumn{1}{|c|}{9}              & \multicolumn{1}{c|}{2024-05-20}  & 2024-05-26  & - Napisanie i przeprowadzenie testów aplikacji                                                                                                                  \\ \hline
        \multicolumn{1}{|c|}{10}             & \multicolumn{1}{c|}{2024-05-27}  & 2024-06-02  & - Naprawianie błędów                                                                                                                                            \\ \hline
        \multicolumn{1}{|c|}{11}             & \multicolumn{1}{c|}{2024-06-03}  & 2024-06-09  & - Wdrażanie poprawek                                                                                                                                            \\ \hline
    \end{tabular}
    \caption{Tabela przedstawiająca harmonogram pracy na przestrzeni kolejnych tygodni}
    \label{tab:HARMONOGRAM}
\end{table}

    \subsubsection*{Tydzień 1.}
    Pierwszy tydzień zostanie poświęcony na poszukiwanie pomysłów do realizacji w ramach projektu. 
    Po wybraniu kandydatów rozważymy narzędzia i technologie, które zostaną użyte w projekcie.

    \subsubsection*{Tydzień 2.}
    W drugim tygodniu uszczegółowimy założenia projektowe.
    \begin{itemize}
        \item Wybierzemy jakie dokładnie obiekty będą identyfikowane przez nasz system.
        \item Ustalimy w jak będzie informować kierowce o wykrytych obiektach.
        \item Rozważymy strukture aplikacji.
        \item Ustalimy koncept wyglądu aplikacji.
    \end{itemize}
    Następnie przystąpimy do tworzenia dokumentacji.

    \subsubsection*{Tydzień 3.}
    Trzeci tydzień poświęcimy na naukę obsługi wybranych narzędzi.

    \subsubsection*{Tydzień 4., 5.}
    W czwartym i piątym tygodniu skupimy się na produkcji modelu zdolnego do wykrywania 
    interesujących nas obiektów. Pod koniec tego etapu będziemy mieli prostą aplikację
    wyświetlającą nagranie wideo z alertami o wykrytych obiektach.

    \subsubsection*{Tydzień 6.}
    Szósty tydzień poświęcimy na stworzenie szkieletu aplikacji.
    Będzie ona bliska wyglądem docelowej wersji i będzie miała zaimplementowany
    model wykrywający obiekty.

    \subsubsection*{Tydzień 7.}
    W siódmym tygodniu dodamy do aplikacji możliwość rysowanie linii pomocniczych.

    \subsubsection*{Tydzień 8.}
    Podczas ósmego tygodnia zajmiemy się intuicyjnością i wygodą użytkowania aplikacji.
    Dokonamy szlifów interfejsu użytkownika.
    
    \subsubsection*{Tydzień 9.}
    W dziewiątym tygodniu przeprowadzimy testy aplikacji.
    
    \subsubsection*{Tydzień 10.}
    Dziesiąty tydzień poświęcimy na naprawianie błędów wykrytych podczas poprzedniego tygodnia.

    \subsubsection*{Tydzień 11.}
    Ostatni tydzień poświęcimy na wdrażanie poprawek zleconych przez klienta. % XD -> jakieś rysuneczki albo lista i fajrant

    \section{Plan testów systemu}
    \subsection{Testy systemu}
Testowane będą funkcjonalności systemu wyszczególnione w sekcji \ref{funkcjonalne}. \newline
Testy będą przeprowadzane przez testera w sposób manualny.

\subsection{Opis testowania funkcjonalności}
\begin{itemize}
	\item \textbf{Wykrywanie pojazdów} - Po poprawnym skonfigurowaniu systemu, tester uruchomi przykładowy materiał wideo. W trakcie oglądania go będzie obserwował czy pojazdy na nagraniu są poprawnie wykrywane. 
	\item \textbf{Wykrywanie pieszych} - Po poprawnym skonfigurowaniu systemu, tester uruchomi przykładowy materiał wideo. W trakcie oglądania go będzie obserwował czy piesi na nagraniu są poprawnie wykrywani.
	\item \textbf{Wykrywanie ostrzegawczych znaków drogowych} - Po poprawnym skonfigurowaniu systemu, tester uruchomi przykładowy materiał wideo. W trakcie oglądania go będzie obserwował czy znaki drogowe na nagraniu są poprawnie wykrywane.
	\item \textbf{Alerty} - Tester, przy użyciu programu konfiguracyjnego, będzie po kolei wybierał wszystkie dostępne rodzaje alertów a następnie oglądając materiał testowy, będzie sprawdzał czy są one poprawnie uruchamiane.
	\item \textbf{Rysowanie linii} - Tester zaobserwuje, czy linie pomocnicze są nakładane na nagranie zgodnie z ustawieniami wybranymi w programie konfiguracyjnym.
	\item \textbf{Konfigurowalne parametry} Tester, wprowadzając zmiany w programie konfiguracyjnym, będzie sprawdzał czy są one poprawnie zapisywane.
\end{itemize}


\subsection{Źródła materiałów wideo}
Materiały wideo na których przeprowadzane będą testy, będą pochodzić z własnych nagrań oraz z internetu. \newline
Kilka przykładowych nagrań z serwisu YouTube:
\begin{itemize}
	\item \url{https://www.youtube.com/watch?v=PNDWr5FgBA8}
	\item \url{https://www.youtube.com/watch?v=DF2d4GqSNQ4}
	\item \url{https://www.youtube.com/watch?v=WC0GY7T3ZlM}
\end{itemize} % 0????


    
\end{document}
