\subsection{Opis testów}
Testowanie projektu polegało na manualnej weryfikacji skuteczności wykrywania kolejnych obiektów na
nagraniach.

Testy zostały przeprowadzone na czterech nagraniach, kolejno: \textit{1.mp4}, \textit{2.mp4}, \textit{3.mp4}
oraz \textit{4.mp4}.
W każdym z nagraniach znajdowało się kilka obiektów, które miały zostać wykryte przez system.

% TODO - uncomment if implemented in settings
% Przygotowany przez nas system umożliwia dopasowanie parametrow wykrywania przez użytkownika. Najbardziej
% znaczące są parametry \textit{confidence\_threshold}, które określają minimalne prawdopodobieństwo, z jakim
% obiekt musi zostać określony, aby został uznany za wykryty. Dla modelu wykrywajacego samochody, znaki STOP i
% ludzi miał wartość 0.7, a dla modeli wykrywających znaki ostrzegawcze: 0.6. Dobór wartości tych parametrów
% pozwala w znaczący sposób wpłynąć na skuteczność wykrywania obiektów.

\subsection{Nagranie 1.mp4}

\subsubsection*{Testowane obiekty i ocena działania systemu}
\begin{itemize}
    \item \textbf{Samochody}
          \begin{itemize}
              \item System poprawnie wykrył wszystkie samochody i ciężarówki, znajdujące się na potencjalnie kolizyjnej trasie.
          \end{itemize}
    \item \textbf{Znak STOP}
          \begin{itemize}
              \item Znak został wykryty przez system.
          \end{itemize}
\end{itemize}

\subsubsection*{Błędne wykrycia}
\begin{itemize}
    \item W nagraniu pojawia się błędna klasyfikacja fragmentu drogi jako samochodu.
    \item Błędne wykrycie znaku odwróconego tyłem jako znak STOP.
\end{itemize}


\subsection{Nagranie 2.mp4}

\subsubsection*{Testowane obiekty i ocena działania systemu}
\begin{itemize}
    \item \textbf{Znak STOP}
          \begin{itemize}
              \item Znak został poprawnie wykryty.
          \end{itemize}
    \item \textbf{Samochody}
          \begin{itemize}
              \item System poprawnie wykrył samochody będące w bliskiej lub średniej odległości od kamery.
              \item System poprawnie wykrył ciężarówki będące w bliskiej odległości od kamery.
          \end{itemize}
    \item \textbf{Znak ostrzegawczy}
          \begin{itemize}
              \item Znak został poprawnie wykryty.
          \end{itemize}
    \item \textbf{Znak ostrzegawczy (próg zwalniający)}
          \begin{itemize}
              \item Znak został poprawnie wykryty.
          \end{itemize}
    \item \textbf{Znak ostrzegawczy (zwierzyna)}
          \begin{itemize}
              \item Znak został poprawnie wykryty, będąc w bliskiej odległości od kamery.
          \end{itemize}
    \item \textbf{Znak ostrzegawczy (ostre zakręty)}
          \begin{itemize}
              \item Znak został poprawnie wykryty, będąc w bliskiej odległości od kamery.
          \end{itemize}
\end{itemize}

\subsubsection*{Błędne wykrycia}
\begin{itemize}
    \item W nagraniu wystąpiły trzy błędne wykrycia znaku STOP.
\end{itemize}

\subsection{Nagranie 3.mp4}
\subsubsection*{Testowane obiekty i ocena działania systemu}
\begin{itemize}
    \item \textbf{Ludzie}
          \begin{itemize}
              \item System poprawnie wykrywa ludzi.
          \end{itemize}
    \item \textbf{Znak ostrzegawczy (uwaga dzieci)}
          \begin{itemize}
              \item System poprawnie wykrył znak.
          \end{itemize}
    \item \textbf{Znak ostrzegawczy (uwaga światła)}
          \begin{itemize}
              \item System poprawnie wykrył znak.
          \end{itemize}
    \item \textbf{Samochody}
          \begin{itemize}
              \item System poprawnie wykrywa samochody w średniej i bliskiej odległości od kamery.
              \item System poprawnie wykrywa ciężarówki w bliskiej odległości od kamery.
          \end{itemize}
    \item \textbf{Znak ostrzegawczy (ostre zakręty)}
          \begin{itemize}
              \item System poprawnie wykrył znak.
          \end{itemize}
\end{itemize}

\subsubsection*{Błędne wykrycia}
\begin{itemize}
    \item System błędnie zidentyfikował: fragment dachu, plakat wyborczy i znak pierwszeństwa jako znaki ostrzegawcze. Błędne wykrycia były krótkie (rzędu 2-3 klatek).
\end{itemize}

\subsection{Nagranie 4.mp4}
\subsubsection*{Testowane obiekty i ocena działania systemu}
\begin{itemize}
    \item \textbf{Samochody}
          \begin{itemize}
              \item System poprawnie wykrywa samochody w bliskiej odległości od kamery.
          \end{itemize}
    \item \textbf{Ludzie}
          \begin{itemize}
              \item System ma problemy z wykrywaniem ludzi, którzy są w pozycji innej niż stojąca. Człowiek w nagraniu jest wykrywany w pojedynczych klatkach.
          \end{itemize}
\end{itemize}

\subsubsection*{Błędne wykrycia}
\begin{itemize}
    \item Brak
\end{itemize}