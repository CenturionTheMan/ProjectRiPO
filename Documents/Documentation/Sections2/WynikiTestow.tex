\subsection{Opis testów}
Testowanie projektu polegało na manualnej weryfikacji skuteczności wykrywania kolejnych obiektów na
nagraniach.

Testy zostały przeprowadzone na czterech nagraniach, kolejno: \textit{1.mp4}, \textit{2.mp4}, \textit{3.mp4}
oraz \textit{4.mp4}.
W każdym z nagraniach znajdowało się kilka obiektów, które miały zostać wykryte przez system.
 
Przygotowany przez nas system umożliwia dopasowanie parametrow wykrywania przez użytkownika. Najbardziej
znaczące są parametry \textit{confidence_threshold}, które określają minimalne prawdopodobieństwo, z jakim
obiekt musi zostać określony, aby został uznany za wykryty. Dla modelu wykrywajacego samochody, znaki STOP i
ludzi miał wartość 0.7, a dla modeli wykrywających znaki ostrzegawcze: 0.6. Dobór wartości tych parametrów 
pozwala w znaczący sposób wpłynąć na skuteczność wykrywania obiektów.

\subsection{Nagranie 1.mp4}

\begin{table}[H]
    \centering
    \begin{tabular}{|c|c|c|}
        \hline
        \textbf{Obiekty do wykrycia}                                                       & \textbf{Ocena działania systemu}                                                                                                                       & \textbf{Błędne wykrycia}                                                                                                                                                                                                    \\ \hline
        Samochody                                                                          & \begin{tabular}[c]{@{}c@{}}System poprawnie wykrył wszystkie samochody i ciężarówki, \\ znajdujące się na potencjalnie kolizyjnej trasie.\end{tabular} & \multirow{2}{*}{\begin{tabular}[c]{@{}c@{}}- W nagraniu przez dwie klatki pojawia się błędna \\ klasyfikacja fragmentu drogi jako samochodu.\\ \\ - Błędne wykrycie znaku odwróconego tyłem \\ jako znak STOP\end{tabular}} \\ \cline{1-2}
        \begin{tabular}[c]{@{}c@{}}Znak ostrzegawczy \\ (ustąp pierwszeństwa)\end{tabular} & Znak został wykryty przez system.                                                                                                                      &                                                                                                                                                                                                                             \\ \hline
    \end{tabular}
    \caption{Wyniki testów dla nagrania 1.mp4}
    % \label{tab:}
\end{table}

\subsection{Nagranie 2.mp4}
\begin{table}[H]
    \centering
    \begin{tabular}{|l|l|l|}
        \hline
        \multicolumn{1}{|c|}{\textbf{Obiekty do wykrycia}} & \multicolumn{1}{c|}{\textbf{Ocena działania systemu}}                                                                                                                                                      & \multicolumn{1}{c|}{\textbf{Błędne wykrycia}}                           \\ \hline
        Znak STOP                                          & Znak został poprawnie wykryty                                                                                                                                                                              & \multirow{6}{*}{- W nagraniu wystąpiły trzy błędne wykrycia znaku STOP} \\ \cline{1-2}
        Samochody                                          & \begin{tabular}[c]{@{}l@{}}System poprawnie wykrył samochody będące w bliskiej lub średniej odległości od kamery.\\ System poprawnie wykrył ciężarówki będące w bliskiej odległości od kamery\end{tabular} &                                                                         \\ \cline{1-2}
        Znak ostrzegawczy                                  & Znak został poprawnie wykryty                                                                                                                                                                              &                                                                         \\ \cline{1-2}
        Znak ostrzegawczy (próg zwalniający)               & Znak został poprawnie wykryty                                                                                                                                                                              &                                                                         \\ \cline{1-2}
        Znak ostrzegawczy (zwierzyna)                      & Znak został poprawnie wykryty, będąc w bliskiej odległości od kamery                                                                                                                                       &                                                                         \\ \cline{1-2}
        Znak ostrzegawczy (ostre zakręty)                  & Znak został poprawnie wykryty, będąc w bliskiej odległości od kamery                                                                                                                                       &                                                                         \\ \hline
    \end{tabular}
    \caption{Wyniki testów dla nagrania 2.mp4}
    % \label{tab:}
\end{table}

\subsection{Nagranie 3.mp4}
\begin{table}[H]
    \centering
    \begin{tabular}{|l|l|l|}
        \hline
        \multicolumn{1}{|c|}{\textbf{Obiekty do wykrycia}} & \multicolumn{1}{c|}{\textbf{Ocena działania systemu}}                                                                                                                                           & \multicolumn{1}{c|}{\textbf{Błędne wykrycia}}                                                                                                                                                                               \\ \hline
        Ludzie                                             & System poprawnie wykrywa ludzi                                                                                                                                                                  & \multirow{5}{*}{\begin{tabular}[c]{@{}l@{}}- System błędnie zidentyfikował: fragment dachu, plakat wyborczy i znak pierwszeństwa \\ jako znaki ostrzegawcze. Błędne wykrycia były krótkie (rzędu 2-3 klatek).\end{tabular}} \\ \cline{1-2}
        Znak ostrzegawczy (uwaga dzieci)                   & System poprawnie wykrył znak                                                                                                                                                                    &                                                                                                                                                                                                                             \\ \cline{1-2}
        Znak ostrzegawczy (uwaga światła)                  & System poprawnie wykrył znak                                                                                                                                                                    &                                                                                                                                                                                                                             \\ \cline{1-2}
        Samochody                                          & \begin{tabular}[c]{@{}l@{}}- System poprawnie wykrywa samochody w średniej i bliskiej odległości od kamery\\ - System poprawnie wykrywa ciężarówki w bliskiej odległości od kamery\end{tabular} &                                                                                                                                                                                                                             \\ \cline{1-2}
        Znak ostrzegawczy (ostre zakręty)                  & - System poprawnie wykrył znak                                                                                                                                                                  &                                                                                                                                                                                                                             \\ \hline
    \end{tabular}
    \caption{Wyniki testów dla nagrania 3.mp4}
    % \label{tab:}
\end{table}

\subsection{Nagranie 4.mp4}
\begin{table}[H]
    \centering
    \begin{tabular}{|l|l|l|}
        \hline
        \multicolumn{1}{|c|}{\textbf{Obiekty do wykrycia}} & \multicolumn{1}{c|}{\textbf{Ocena działania systemu}}                                                                                                                                & \multicolumn{1}{c|}{\textbf{Błędne wykrycia}} \\ \hline
        Samochody                                          & System wykrywa samochody położone blisko kamery                                                                                                                                      & \multirow{2}{*}{Brak}                         \\ \cline{1-2}
        Ludzie                                             & \begin{tabular}[c]{@{}l@{}}System ma problemy z wykrywaniem ludzi, którzy są w pozycji innej niż stojąca.\\ Człowiek w nagraniu jest wykrywany w pojedynczych klatkach.\end{tabular} &                                               \\ \hline
    \end{tabular}
    \caption{Wyniki testów dla nagrania 4.mp4}
    % \label{tab:}
\end{table}