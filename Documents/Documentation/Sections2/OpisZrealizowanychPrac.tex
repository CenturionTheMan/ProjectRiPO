\subsection*{Początki}

Projekt rozpoczęliśmy od dokładnej analizy dostępnych rozwiązań stosowanych w procesie rozpoznawania obrazów. Zidentyfikowaliśmy kilka potencjalnych narzędzi i bibliotek, które mogłyby być użyteczne w naszym projekcie:
\begin{itemize}
    \item OpenCV - biblioteka do przetwarzania obrazu, ze szczególnym uwzględnieniem modułu \textit{Haar Cascades}, który pozwala na wykrywanie obiektów.
    \item Model YOLO - zaawansowany model sieci neuronowej do wykrywania obiektów.
\end{itemize}

Na początku testowaliśmy \textit{Haar Cascades}, jednak napotkaliśmy kilka istotnych problemów:
\begin{enumerate}
    \item Publicznie dostępne konfiguracje modułu słabo wykrywały samochody, ludzi i znaki, czyli obiekty, które miały być identyfikowane w ramach naszego projektu.
    \item Ręczne stworzenie pliku konfiguracyjnego było czasochłonne.
    \item Podejście to słabo radziło sobie z wykrywaniem obiektów na nagraniach innych niż te, na których model był trenowany.
\end{enumerate}

W związku z powyższymi problemami zdecydowaliśmy się porzucić \textit{Haar Cascades} na rzecz pretrenowanego
modelu YOLOv5.

\subsection*{YOLOv5}

Model YOLOv5 \cite[zobacz:][]{YOLOv5} to pretrenowana sieć neuronowa przeznaczona do wykrywania obiektów na obrazach. Udostępniona wersja potrafi wykrywać samochody, ludzi i znaki stop, co pokrywało większość obiektów, które mieliśmy identyfikować.

Jednakże, przetwarzanie obrazu wideo na żywo okazało się bardzo wymagające obliczeniowo. W bazowej wersji przetwarzanie obrazu było zbyt wolne, co skutkowało niepłynnością wyświetlania. Problem ten rozwiązaliśmy poprzez uruchomienie modelu na karcie graficznej oraz zmniejszenie częstotliwości analizy obrazu - wykrywaliśmy obiekty co drugą klatkę, co pozwoliło na znaczną poprawę wydajności.

\subsection*{Roboflow}

Kolejnym krokiem było zaimplementowanie wykrywania znaków ostrzegawczych.
W tym celu wykorzystaliśmy model \textit{znaki-drogowe-w-polsce} \citep[zobacz:][]{ZnakiWPolsce}, który jest wytrenowany do wykrywania polskich znaków drogowych. Model ten również uruchomiliśmy na GPU, co pozwoliło na efektywne wykrywanie tych znaków w naszym systemie.

\subsection*{Linie}

Następnie przystąpiliśmy do implementacji rysowania linii na obrazie. Zadanie to zrealizowaliśmy za pomocą biblioteki OpenCV, która oferuje szeroki zakres narzędzi do manipulacji obrazami.

\subsection*{Interfejs}

Na zakończenie stworzyliśmy interfejs użytkownika, który umożliwia interakcję z programem. Interfejs ten został zaprojektowany tak, aby był intuicyjny i łatwy w obsłudze, co pozwala użytkownikom na pełne wykorzystanie funkcjonalności naszego systemu.
