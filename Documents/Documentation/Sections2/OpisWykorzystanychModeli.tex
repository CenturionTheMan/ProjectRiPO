\subsection*{O wykorzystanych modelach}
W projekcie oparlismy mechanizm wykrywania obiektów na modelach wykorzystujących głębokie sieci neuronowe. 
W tym celu wykorzystaliśmy architekturę YOLOv5 (w wariancie 's' - najmniejszym i najmniej kosztownym
obliczeniowo), 
która pozwala na wykrywanie obiektów w czasie rzeczywistym. Dodatkowo 

Model wykorzystaliśmy dwojako. 
Do wykrywania ludzi, samochodow i znaków stop wykorzystujemy wariacje wytrenowaną na zbiorze danych
\textit{COCO} \citep[zobacz:][]{COCO}. Natomiast do wykrywania znaków ostrzegawczych wykorzystaliśmy model
\textit{znaki-drogowe-w-polsce} \citep[zobacz:][]{ZnakiWPolsce}, czyli YOLOv5 wytrenowany na zbiorze zdjęć
polskich znaków drogowych.

\subsection*{Zbiór danych COCO}
Zbiór danych \textit{COCO} to jeden z najpopularniejszych zbiorów danych do trenowania modeli wykrywania
obiektów.
Zawiera on ponad 330 tysięcy obrazów (z tego 200 tysiecy dostosowane do wykrywania obiektów) 
z ponad 80 kategoriami obiektów.

\subsection*{Znaki drogowe w Polsce}
Znaki drogowe w Polsce \citep[zobacz:][]{ZnakiWPolsce} to wytrenowany na zbiorze zdjęć polskich znaków drogowych model YOLOv5.
Zbiór danych zawiera 1687 zdjęć z 18 kategoriami znaków drogowych w tym.