    \subsection{Opis projektu i systemu}
    Projekt ma na celu opracowanie systemu obsługującego kamerę samochodową, wyposażoną 
    w funkcję wykrywania określonych obiektów, z możliwością konfiguracji jej parametrów. 
    Głównym zadaniem systemu jest analiza obrazu z kamery oraz sygnalizacja użytkownikowi wykrycia 
    obiektu spełniającego określone kryteria. 
    W ramach projektu przewiduje się wykrywanie takich obiektów jak: [???], [???], [???].

    W skład systemu wchodzi również edytor umożliwiający konfigurację oprogramowania wykonawczego. 
    Edytor ten umożliwia kadrowanie obrazu, edycję linii, wybór obiektów do wykrywania oraz ustawianie 
    parametrów wykrywania obiektów. Dodatkowo, użytkownik może ustawić metodę informowania o wykryciu obiektów, 
    taką jak ramka wokół obiektu, alert na ekranie lub alert dźwiękowy.

    System zapewnia obsługę zarówno nagrania wideo z kamery, jak i obrazu w czasie rzeczywistym. 
    Dzięki temu użytkownik ma możliwość monitorowania otoczenia pojazdu w czasie rzeczywistym oraz analizy
    wcześniej zarejestrowanych nagrań. 

    \subsection{Analiza istniejących rozwiązań} %TODO
        \subsubsection*{FineVu GX1000}
        \begin{itemize}
            \item Zalety
            \begin{itemize}
                \item Wysoka rozdzielczość kamer: Quad HD + Quad HD
                \item Szeroki kąt widzenia: $122^\circ / 122^\circ$
                \item Funkcja detekcji ruchu 
                \item Ostrzeżenia o fotoradarach, kamerach średniej prędkości i innych kontrolach drogowych
                \item Auto Night Vision - tryb nocny i automatyczna poprawa obrazu w dzień
            \end{itemize}
            \item Wady
            \begin{itemize}
                \item Wysoka cena: 1349 zł lub 1399 zł w zaleznosci od rozmiaru karty pamięci
                \item Ryzyko przegrzania: W niektórych warunkach klimatycznych lub przy długotrwałym 
                użytkowaniu istnieje ryzyko przegrzewania się urządzenia, co może wpłynąć na jego wydajność.
                \item Brak funkcji wykrywania statycznych obiektów (pachołki, krawężniki itp.)
            \end{itemize}
        \end{itemize}

        Link: \textit{\url{https://finevu.pl/kamery-samochodowe/finevu-gx1000/}}
        

        \subsubsection*{M3S/M6S Infrared Night Vision Camera for Cars} %! Trzeba to sprawdzić jeszcze bo dziwne
        \begin{itemize}
            \item Zalety
            \begin{itemize}
                \item Technologia termowizyjna: umożliwia on użytkownikom obserwację otoczenia pojazdu 
                w warunkach słabego oświetlenia i nawet w całkowitej ciemności.
                \item Wykrywanie obiektów: Produkt może być używany do wykrywania obiektów na drodze, 
                włączając w to zwierzęta, osoby poruszające się pieszo czy inne pojazdy, 
                co może pomóc w unikaniu kolizji.
            \end{itemize}
            \item Wady
            \begin{itemize}
                \item Niska dostępność
            \end{itemize}
        \end{itemize}

        Link: \textit{https://www.infiray.com/products/m6s-infrared-night-vision-camera-for-cars.html}
