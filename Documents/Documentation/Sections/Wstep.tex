    \subsection{Opis projektu i systemu}
    Projekt ma na celu opracowanie systemu obsługującego kamerę samochodową, wyposażoną 
    w funkcję wykrywania określonych obiektów, z możliwością konfiguracji jej parametrów. 
    Głównym zadaniem systemu jest analiza obrazu z kamery oraz sygnalizacja użytkownikowi wykrycia 
    obiektu spełniającego określone kryteria. 
    W ramach projektu przewiduje się wykrywanie takich obiektów jak: [???], [???], [???].

    W skład systemu wchodzi również edytor umożliwiający konfigurację oprogramowania wykonawczego. 
    Edytor ten umożliwia kadrowanie obrazu, edycję linii, wybór obiektów do wykrywania oraz ustawianie 
    parametrów wykrywania obiektów. Dodatkowo, użytkownik może ustawić metodę informowania o wykryciu obiektów, 
    taką jak ramka wokół obiektu, alert na ekranie lub alert dźwiękowy.

    System zapewnia obsługę zarówno nagrania wideo z kamery, jak i obrazu w czasie rzeczywistym. 
    Dzięki temu użytkownik ma możliwość monitorowania otoczenia pojazdu w czasie rzeczywistym oraz analizy
    wcześniej zarejestrowanych nagrań. 

    \subsection{Analiza istniejących systemów} %TODO
    W ramach tego punktu należy wykonać tzw. analizę stanu techniki - czyli wskazać istniejące systemy i odnieść się do ich wad i zalet.