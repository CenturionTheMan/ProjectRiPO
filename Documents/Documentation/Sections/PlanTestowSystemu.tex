\subsection{Testy systemu}
Testowane będą funkcjonalności systemu wyszczególnione w sekcji \ref{funkcjonalne}. \newline
Testy będą przeprowadzane przez testera w sposób manualny.

\subsection{Opis testowania funkcjonalności}
\begin{itemize}
	\item \textbf{Wykrywanie pojazdów} - Po poprawnym skonfigurowaniu systemu, tester uruchomi przykładowy materiał wideo. W trakcie oglądania go będzie obserwował czy pojazdy na nagraniu są poprawnie wykrywane. 
	\item \textbf{Wykrywanie pieszych} - Po poprawnym skonfigurowaniu systemu, tester uruchomi przykładowy materiał wideo. W trakcie oglądania go będzie obserwował czy piesi na nagraniu są poprawnie wykrywani.
	\item \textbf{Wykrywanie ostrzegawczych znaków drogowych} - Po poprawnym skonfigurowaniu systemu, tester uruchomi przykładowy materiał wideo. W trakcie oglądania go będzie obserwował czy znaki drogowe na nagraniu są poprawnie wykrywane.
	\item \textbf{Alerty} - Tester, przy użyciu programu konfiguracyjnego, będzie po kolei wybierał wszystkie dostępne rodzaje alertów a następnie oglądając materiał testowy, będzie sprawdzał czy są one poprawnie uruchamiane.
	\item \textbf{Rysowanie linii} - Tester zaobserwuje, czy linie pomocnicze są nakładane na nagranie zgodnie z ustawieniami wybranymi w programie konfiguracyjnym.
	\item \textbf{Konfigurowalne parametry} Tester, wprowadzając zmiany w programie konfiguracyjnym, będzie sprawdzał czy są one poprawnie zapisywane.
\end{itemize}


\subsection{Materiały wideo}
Razem z programem udostępnione jest 5 nagrań testowych.
\newline
Nagrania 1 do 3 prezentują obraz z przedniej kamery, a nagranie 4 z kamery tylnej

\subsection{Charakterystyka modelu wykrywania}
Działanie systemu:
\begin{itemize}
	\item Poprawnie wykrywane są obiekty znajdujące się w bliskiej odległości od pojazdu, obiekty znajdujące się w większej odległości mogą zostać niewykryte.
	\item Obiekty, które nie są widoczne w całości mogą zostać nierozpoznane.
	\item Obiekty ośnieżone (np. samochody) mogą nie zostać wykryte poprawnie
	\item Istnieje niewielka możliwość niepoprawnego wykrycia np. odwróconego znaku jako znaku stop
\end{itemize}

Funkcjonowanie systemu na przykładzie poszczególnych nagrań.
\newline

Nagranie 1: \newline
Na tym nagraniu wykryte zostaną:
\begin{itemize}
	\item Samochody - najlepiej wykrywane są, kiedy znajdują się dość blisko kamery i nie są częściowo zasłonięte
	\item znak ostrzegawczy
\end{itemize}

Nagranie 2: \newline
Na tym nagraniu wykryte zostaną:
\begin{itemize}
	\item Samochody - niektóre samochody nadjeżdżające z przeciwka są wykrywane z bardzo bliskiej odległości
	\item znak stop
	\item znaki ostrzegawcze
\end{itemize}


Nagranie 3: \newline
Na tym nagraniu wykryte zostaną:
\begin{itemize}
	\item Samochody
	\item Piesi - osoby znajdujące się w większej odległości i na bokach nagrania są słabo wykrywane
	\item znaki ostrzegawcze
\end{itemize}

Nagranie 4: \newline
Na tym nagraniu wykryte zostaną:
\begin{itemize}
	\item Samochody
\end{itemize}

