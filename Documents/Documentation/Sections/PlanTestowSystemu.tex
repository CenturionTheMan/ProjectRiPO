\subsection{Testy systemu}
Testowane będą funkcjonalności systemu wyszczególnione w sekcji \ref{funkcjonalne}. \newline
Testy będą przeprowadzane przez testera w sposób manualny.

\subsection{Opis testowania funkcjonalności}
\begin{itemize}
	\item \textbf{Wykrywanie pojazdów} - Po poprawnym skonfigurowaniu systemu, tester uruchomi przykładowy materiał wideo. W trakcie oglądania go będzie obserwował czy pojazdy na nagraniu są poprawnie wykrywane. 
	\item \textbf{Wykrywanie pieszych} - Po poprawnym skonfigurowaniu systemu, tester uruchomi przykładowy materiał wideo. W trakcie oglądania go będzie obserwował czy piesi na nagraniu są poprawnie wykrywani.
	\item \textbf{Wykrywanie ostrzegawczych znaków drogowych} - Po poprawnym skonfigurowaniu systemu, tester uruchomi przykładowy materiał wideo. W trakcie oglądania go będzie obserwował czy znaki drogowe na nagraniu są poprawnie wykrywane.
	\item \textbf{Alerty} - Tester, przy użyciu programu konfiguracyjnego, będzie po kolei wybierał wszystkie dostępne rodzaje alertów a następnie oglądając materiał testowy, będzie sprawdzał czy są one poprawnie uruchamiane.
	\item \textbf{Rysowanie linii} - Tester zaobserwuje, czy linie pomocnicze są nakładane na nagranie zgodnie z ustawieniami wybranymi w programie konfiguracyjnym.
	\item \textbf{Konfigurowalne parametry} Tester, wprowadzając zmiany w programie konfiguracyjnym, będzie sprawdzał czy są one poprawnie zapisywane.
\end{itemize}


\subsection{Źródła materiałów wideo}
Materiały wideo na których przeprowadzane będą testy, będą pochodzić z własnych nagrań oraz z internetu. \newline
Kilka przykładowych nagrań z serwisu YouTube:
\begin{itemize}
	\item \url{https://www.youtube.com/watch?v=PNDWr5FgBA8}
	\item \url{https://www.youtube.com/watch?v=DF2d4GqSNQ4}
	\item \url{https://www.youtube.com/watch?v=WC0GY7T3ZlM}
\end{itemize}