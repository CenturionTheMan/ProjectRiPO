
\begin{table}[H]
    \centering
    \begin{tabular}{|ccc|l|}
        \hline
        \multicolumn{3}{|c|}{\textbf{Tydzień}}                                                & \multicolumn{1}{c|}{\multirow{2}{*}{\textbf{Zadania}}}                                                                                                          \\ \cline{1-3}
        \multicolumn{1}{|c|}{\textbf{numer}} & \multicolumn{1}{c|}{\textbf{od}} & \textbf{do} & \multicolumn{1}{c|}{}                                                                                                                                           \\ \hline
        \multicolumn{1}{|c|}{1}              & \multicolumn{1}{c|}{2024-03-25}  & 2024-03-31  & \begin{tabular}[c]{@{}l@{}}- Omówienie   tematu \\ - Poszukiwanie pomysłów do realizacji w ramach projektu. \\ - Wybór narzędzi   i technologii\end{tabular}   \\ \hline
        \multicolumn{1}{|c|}{2}              & \multicolumn{1}{c|}{2024-04-01}  & 2024-04-07  & \begin{tabular}[c]{@{}l@{}}- Uszczegółowianie założeń projektowych\\ - Tworzenie dokumentacji.\end{tabular}                                                     \\ \hline
        \multicolumn{1}{|c|}{3}              & \multicolumn{1}{c|}{2024-04-08}  & 2024-04-14  & - Nauka obsługi wybranych narzędzi.                                                                                                                             \\ \hline
        \multicolumn{1}{|c|}{4}              & \multicolumn{1}{c|}{2024-04-15}  & 2024-04-21  & - Produkcja modelu zdolnego do wykrywania obiektów                                                                                     \\ \hline
        \multicolumn{1}{|c|}{5}              & \multicolumn{1}{c|}{2024-04-22}  & 2024-04-28  & - Produkcja modelu zdolnego do wykrywania obiektów                                                                                  \\ \hline
        \multicolumn{1}{|c|}{6}              & \multicolumn{1}{c|}{2024-04-29}  & 2024-05-05  & - Stworzenie szkieletu aplikacji z funkcjonalnością wykrywania obiektów                                                                                         \\ \hline
        \multicolumn{1}{|c|}{7}              & \multicolumn{1}{c|}{2024-05-06}  & 2024-05-12  & - Implementacja funkcjonalności rysowania linii pomocniczych                                                                                                    \\ \hline
        \multicolumn{1}{|c|}{8}              & \multicolumn{1}{c|}{2024-05-13}  & 2024-05-19  & \begin{tabular}[c]{@{}l@{}}- Ogólna weryfikacja spełnienia założeń projektowych.\\ - Kontrola jakości i poprawki związane z użytkowaniem aplikacji\end{tabular} \\ \hline
        \multicolumn{1}{|c|}{9}              & \multicolumn{1}{c|}{2024-05-20}  & 2024-05-26  & - Napisanie i przeprowadzenie testów aplikacji                                                                                                                  \\ \hline
        \multicolumn{1}{|c|}{10}             & \multicolumn{1}{c|}{2024-05-27}  & 2024-06-02  & - Naprawianie błędów                                                                                                                                            \\ \hline
        \multicolumn{1}{|c|}{11}             & \multicolumn{1}{c|}{2024-06-03}  & 2024-06-09  & - Wdrażanie poprawek                                                                                                                                            \\ \hline
    \end{tabular}
    \caption{Tabela przedstawiająca harmonogram pracy na przestrzeni kolejnych tygodni}
    \label{tab:HARMONOGRAM}
\end{table}

    \subsubsection*{Tydzień 1.}
    Pierwszy tydzień zostanie poświęcony na poszukiwanie pomysłów do realizacji w ramach projektu. 
    Po wybraniu kandydatów rozważymy narzędzia i technologie, które zostaną użyte w projekcie.

    \subsubsection*{Tydzień 2.}
    W drugim tygodniu uszczegółowimy założenia projektowe.
    \begin{itemize}
        \item Wybierzemy jakie dokładnie obiekty będą identyfikowane przez nasz system.
        \item Ustalimy w jak będzie informować kierowce o wykrytych obiektach.
        \item Rozważymy strukture aplikacji.
        \item Ustalimy koncept wyglądu aplikacji.
    \end{itemize}
    Następnie przystąpimy do tworzenia dokumentacji.

    \subsubsection*{Tydzień 3.}
    Trzeci tydzień poświęcimy na naukę obsługi wybranych narzędzi.

    \subsubsection*{Tydzień 4., 5.}
    W czwartym i piątym tygodniu skupimy się na produkcji modelu zdolnego do wykrywania 
    interesujących nas obiektów. Pod koniec tego etapu będziemy mieli prostą aplikację
    wyświetlającą nagranie wideo z alertami o wykrytych obiektach.

    \subsubsection*{Tydzień 6.}
    Szósty tydzień poświęcimy na stworzenie szkieletu aplikacji.
    Będzie ona bliska wyglądem docelowej wersji i będzie miała zaimplementowany
    model wykrywający obiekty.

    \subsubsection*{Tydzień 7.}
    W siódmym tygodniu dodamy do aplikacji możliwość rysowanie linii pomocniczych.

    \subsubsection*{Tydzień 8.}
    Podczas ósmego tygodnia zajmiemy się intuicyjnością i wygodą użytkowania aplikacji.
    Dokonamy szlifów interfejsu użytkownika.
    
    \subsubsection*{Tydzień 9.}
    W dziewiątym tygodniu przeprowadzimy testy aplikacji.
    
    \subsubsection*{Tydzień 10.}
    Dziesiąty tydzień poświęcimy na naprawianie błędów wykrytych podczas poprzedniego tygodnia.

    \subsubsection*{Tydzień 11.}
    Ostatni tydzień poświęcimy na wdrażanie poprawek zleconych przez klienta.