W niniejszym punkcie przedstawiamy listę wymagań technicznych dla tworzonego systemu. 
Przed rozpoczęciem implementacji oprogramowania kluczowe jest dokładne określenie środowiska, 
identyfikacja kluczowych komponentów technologicznych oraz uwzględnienie aspektów związanych 
z systemem operacyjnym, infrastrukturą sprzętową, bezpieczeństwem danych oraz wyborem 
technologii programistycznych.

\subsubsection*{System Operacyjny}
Oprogramowanie będzie kompatybilne z systemem operacyjnym Windows.

\subsubsection*{Wymagania Sprzętowe} %! To do research'u jeszcze
\begin{itemize}
    \item Procesor: Intel Core i5 lub równoważny
    \item Pamięć RAM: Minimum 8GB
    \item Karta graficzna: Zintegrowana lub dedykowana wspierająca OpenGL %?????
    \item System operacyjny: Windows 7 lub nowszy
\end{itemize}

\subsubsection*{Podstawowe Predykcje Wydajnościowe}
Oprogramowanie powinno działać płynnie na wymaganym sprzęcie, zapewniając responsywność 
interfejsu użytkownika oraz szybką obróbkę danych w czasie rzeczywistym.

\subsubsection*{Kwestie Sieciowe i Bezpieczeństwa}
\begin{itemize}
    \item Oprogramowanie będzie działać w trybie lokalnym, nie wymagając połączenia z internetem.
    \item System Windows będzie wyposażony w aktualną ochronę antywirusową w celu zapewnienia bezpieczeńwa danych.
\end{itemize}

\subsubsection*{Wymagania dotyczące Kamer}
\begin{itemize}
    \item Oprogramowanie będzie kompatybilne z różnymi modelami kamer kompatybilnymi z systemem Windows.
    \item Wsparcie dla standardowych rozdzielczości obrazu oraz prędkości klatek.
\end{itemize}

\subsubsection*{Język Programowania i Biblioteki}
\begin{itemize}
    \item Oprogramowanie będzie napisane w języku Python.
    \item Do przetwarzania obrazu będzie wykorzystywana biblioteka OpenCV.
    \item Interfejs zostanie zaimplementowany przy użyciu biblioteki TKinter.
\end{itemize}

\subsubsection*{Konteneryzacja}
Oprogramowanie zostanie skonteneryzowane przy użyciu narzędzia Docker, 
aby zapewnić łatwość w wdrażaniu oraz zarządzaniu środowiskiem aplikacyjnym.